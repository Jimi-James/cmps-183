\documentclass[24pt]{beamer}
\usetheme{CambridgeUS}

\title{M$\mu$tor}
\author{Jimi Bove jebove@ucsc.edu, Darion Toney dtoney@ucsc.edu}
\date{\today}

\begin{document}
\maketitle

\begin{frame}
\frametitle{Original Proposal}
\begin{itemize}
  \item Music tutor website
  \item Inspired by LenMus and Codecademy
  \item Music lessons \& practice, all in Javascript
  \item "Practice" means interactive sections and minigames
  \item Saved progress by user account
\end{itemize}
\end{frame}

\begin{frame}
\frametitle{Missing Features}
\begin{itemize}
  \item Most lessons
  \item All interactive practice (didn't even have time to start that)
  \item Saved progress
\end{itemize}
\end{frame}

\begin{frame}
\frametitle{Bugs}
\begin{itemize}
  \item Music won't play perfectly in rhythm
  \item Music plays all notes at once after the first playing until refresh
  \item Staff notation might not show up until the link to it is clicked twice, or not show up at all due to a completely inexplicable and unhelpful VexFlow error message
  \item All are bugs with the libraries used for music notation and MIDI playback with Javascript, not caused by us and not present in their own examples (which we used as reference)
  \item Countless other bugs that we've already fixed
\end{itemize}
\end{frame}

\begin{frame}
\frametitle{The Story}
\begin{itemize}
  \item 2 very complex libraries that are the most popular (and possibly only) options for music notation and MIDI playback in Javascript: VexFlow \& MIDI.js
  \item Both libraries seemed solid and had countless great, working examples
  \item Almost no documentation, and no decent documentation at all
  \item Had to read all of the VexFlow and MIDI.js code to learn how to do almost anything
  \item Spent the entire quarter trying to fix these bugs
  \item Technically a 3rd library, VexTab, which is an addon to VexFlow that makes it a little easier to deal with
  \item VexTab at least worked great and had great documentation
\end{itemize}
\end{frame}

\begin{frame}
\frametitle{Now, for the demo!}
\end{frame}

\begin{frame}
\frametitle{Questions?}
\end{frame}
\end{document}
